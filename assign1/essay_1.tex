%%%%%%%%%%%%%%%%%%%%%%%%%%%%%%%%%%%%%%%%%%
% Simple Sectioned Essay Template
% LaTeX Template
% i
% This template has been downloaded from:
% http://www.latextemplates.com
%
% Note:
% The \lipsum[#] commands throughout this template generate dummy text
% to fill the template out. These commands should all be removed when 
% writing essay content.
%
%%%%%%%%%%%%%%%%%%%%%%%%%%%%%%%%%%%%%%%%%

%----------------------------------------------------------------------------------------
%       PACKAGES AND OTHER DOCUMENT CONFIGURATIONS
%----------------------------------------------------------------------------------------
% comentarios para el PROFESOR
%lA PRIMERA MEDIA HORA USARLA PARA INTRODUCIR:
% PYTHON
% DAR UNA PEQUEÑA DEMO DE GIT
% USAR UNA VEZ PYCHARM

\documentclass[12pt]{article} % Default font size is 12pt, it can be changed here
\usepackage[spanish]{babel}
\usepackage[utf8]{inputenc}
\usepackage{listings}
\usepackage{color}
\usepackage{caption}
\usepackage{url}
\usepackage[dvips]{graphicx}
\usepackage{geometry} % Required to change the page size to A4
%\geometry{a4paper} % Set the page size to be A4 as opposed to the default US Letter
\usepackage{framed}

\usepackage{graphicx} % Required for including pictures

\usepackage{float} % Allows putting an [H] in \begin{figure} to specify the exact location of the figure
\usepackage{wrapfig} % Allows in-line images such as the example fish picture

\usepackage{fancyhdr}
\pagestyle{fancy}
\fancyhf{}
\fancyhead[RO]{{Práctica-1}} 
\fancyhead[LO]{Proyecto de Sistemas Informáticos}
%\fancyhead[RO]{{\leftmark}} 
\fancyfoot[LE,RO]{{ \thepage }}

%\usepackage{lipsum} % Used for inserting dummy 'Lorem ipsum' text into the template
\definecolor{grey}{rgb}{0.9,0.9,0.9}

\linespread{1.2} % Line spacing

%\setlength\parindent{0pt} % Uncomment to remove all indentation from paragraphs

\graphicspath{{./Pictures/}} % Specifies the directory where pictures are stored
\newcounter{ejercicioNo}
\begin{document}

%----------------------------------------------------------------------------------------
%       TITLE PAGE
%----------------------------------------------------------------------------------------

\begin{titlepage}

\newcommand{\HRule}{\rule{\linewidth}{0.5mm}} % Defines a new command for the horizontal lines, change thickness here

\center % Center everything on the page

\textsc{\LARGE Universidad Aut\'{o}noma de Madrid}\\[1.5cm] % Name of your university/college
%\textsc{\Large Proyecto de Sistemas Informaticos}\\[0.5cm] % Major heading such as course name
%\textsc{\large Departamento de Informatica}\\[0.5cm] % Minor heading such as course title
\textsc{\Large Departamento de Inform\'{a}tica}\\[0.5cm] % Minor heading such as course title

\HRule \\[0.4cm]
{ \huge \bfseries Proyecto de Sistemas Inform\'{a}ticos\\[0.5cm] Pr\'{a}ctica - 1}\\[0.4cm] % Title of your document
\HRule \\[1.5cm]

%\begin{minipage}{0.4\textwidth}
%\begin{flushleft}
% \large
%\emph{Author:}\\
%Roberto  \textsc{Marabini Ruiz} % Your name
%\end{flushleft}
%\end{minipage}

%\begin{minipage}{0.4\textwidth}
%\begin{flushright} \large
%\emph{Supervisor:} \\
%Dr. James \textsc{Smith} % Supervisor's Name
%\end{flushright}
%\end{minipage}\\[4cm]

%{\large \today}\\[3cm] % Date, change the \today to a set date if you want to be precise

%\includegraphics{Logo}\\[1cm] % Include a department/university logo - this will require the graphicx package

\vfill % Fill the rest of the page with whitespace
%\begin{minipage}{0.4\textwidth}
\begin{flushright}
 \large
%\emph{Author:}\\
Roberto  \textsc{Marabini Ruiz} % Your name
\end{flushright}
%\end{minipage}

\end{titlepage}

%----------------------------------------------------------------------------------------
%       TABLE OF CONTENTS
%----------------------------------------------------------------------------------------

\tableofcontents % Include a table of contents

\newpage % Begins the essay on a new page instead of on the same page as the table of contents 

%----------------------------------------------------------------------------------------
%       OBJETIVOS
%----------------------------------------------------------------------------------------

\section {Objetivos}

El objetivo primordial de esta asignatura es desarrollar una aplicación web usando un entorno de desarrollo profesional. En este curso usaremos Django que es un entorno desarrollado en Python.
Durante la primera pr\'{a}ctica se introducirá el lenguaje Python, el software para control 
de versiones llamado git y el IDE pycharm.

%------------------------------------------------
%\subsection{Introdución} % Sub-section

%Python es un lenguaje interpretado. No hay declaraciones de tipos de variables, parámetros, funciones o métodos. Los ficheros con código Python suelen acabar con la extensión \emph{py}.

%Conectaros a \textit{Google’s Python class} (\url{https://developers.google.com/edu/python/}) y consultar el material destinado al primer d\'{i}a del curso.  Esto es, \texttt{Python Set  Up}, \texttt{Introduction}, \texttt{Strings}, \texttt{Lists}, \texttt{Sorting}, and \texttt{Dict and File pages}. Igualmente debes realizar los ejercicios descritos en los los ficheros \texttt{string1.py}, \texttt{string2.py}, \texttt{list1.py}, \texttt{list2.py} and \texttt{wordcount.py}.

%Como editor (IDE) utilizad \texttt{pycharm}. Durante las practicas te pediremos que entregues proyectos creados en este entorno.

\textbf{Evaluación}: Puesto que la soluci\'on a los problemas propuestos en esta memoria se encuentran disponibles en internet esta práctica será evaluado \'unicamente con la nota APTO o NO APTO. El propósito de esta primera pr\'actica es sencillamente servir como introducción a PYTHON y como medio de auto-evaluar tu nivel programando en dicho lenguaje. En caso de que decidas consultar puntualmente el c\'odigo existente te aconsejamos que no copies 
y pegues partes del c\'odigo sino que las teclees. 

%\textbf{Recomendación}: Debéis realizar una única entrega por grupo pero en esta primera práctica os 
%aconsejamos que trabajéis  de forma independiente y cada miembro del equipo realice todos los ejercicios propuesto.
%\begin{lstlisting} [frame=single,language=java,lineskip={-1.0pt},breaklines=true] 
%##!/usr/bin/env python
%print 'Hola'
%\end{lstlisting}
%\section{Pycharm}
%PyCharm es un IDE (Entorno de desarrollo integrado) desarrollado específicacmente para Python.
%Recomendamos su uso en esta asignatura 

\section{Trabajo a Realizar Durante la Primera Semana de la práctica}
Conectaros a \textit{Google’s Python class} (\url{https://developers.google.com/edu/python/}) y consultar  las secciones tituladas \texttt{Python Set  Up}, \texttt{Introduction} y  \texttt{Strings}. Tras realizar los ejemplos propuestos implementad  los ejercicios descritos en los los ficheros \texttt{string1.py} y \texttt{string2.py} (están en el link \url{https://developers.google.com/edu/python/google-python-exercises.zip}). Acostumbraros a ir guardando el código generado en un repositorio gestionado usando git.

\section{Trabajo a Entregar Durante la Primera Semana de la práctica}

\begin{minipage}{\linewidth}
\begin{framed}
\addtocounter{ejercicioNo}{1} 
Ejercicio \arabic{ejercicioNo}:
%\label{Ejercicio_1}
\begin{enumerate}
 \item Conectaros a http://bitbucket.org/ y registraros. Vuestro nombre de usuario debe ser la inicial seguida de vuestro apellido siempre que sea posible, si el usuario ya ha sido asignado añadir un n\'umero tras el apellido.
 \item Cread un repositorio llamado psi\_numeroGrupo\_numeroPareja\_p1. El ``Access level'' del repositorio debe ser ``This is a private repository''. Duplicar el repositorio localmente usando el comando \texttt{git clone direccionRepositorio}.

 \item Tras crear el repositorio:
 \begin{itemize}
 \item Modificar los servicios de acceso de forma que los dos miembros de vuestro grupo puedan modificarlo.
 \item El primer miembro del grupo deberá subir el c\'{o}digo necesario para crear la aplicación \texttt{hello.py} (el código se haya disponible en la sección Phyton Set Up the Google python class en el fichero \texttt{google-python-exercises.zip}).
 \item En un ordenador diferente el segundo miembro del grupo (usando su usuario en el repositorio) deberá: (1) bajarse el repositorio, (2) modificar el fichero  para que escriba ``Hola Mundo'' en lugar de ``Hello World''  y (3) subir la modificación al repositorio
   \item El primer miembro del grupo (usando su usuario en el repositorio) deberá: (1) actualizar el repositorio local, (2) crear un \textbf{NUEVO} fichero llamado hail.py  que sea idéntico al fichero hello.py pero cambiando el saludo ``Hola Mundo'' por ``Hallo Welt'' y (3) subir las modificaciones al repositorio.
 \end{itemize}
\item Añadid los ejercicios \texttt{string1.py} y \texttt{string2.py} al repositorio y subidlo a Bitbucket. 
\end{enumerate}
\end{framed}
\end{minipage}\\\\


%------------------------------------------------


%------------------------------------------------
\section{Trabajo a Realizar Durante la Segunda Semana de la práctica}
Conectaros a \textit{Google’s Python class} (\url{https://developers.google.com/edu/python/}) y centraros en las secciones  \texttt{Lists}, \texttt{Sorting} y \texttt{Dict and File pages}. Tras realizar los ejemplos propuestos implementar  los ejercicios descritos en los los ficheros: \texttt{list1.py} y \texttt{wordcount.py}.
Utiliza el fichero \texttt{alice.txt} como entrada del programa wordcount.py.

\section{Material a entregar al finalizar la práctica} % Sub-sub-section

\begin{minipage}{\linewidth}
\begin{framed}
Esta práctica no requiere la entrega de ninguna memoria. Utilizando la aplicación moodle se debe entregar en un único fichero zip:
\begin{enumerate}
\item El proyecto de PyCharm conteniendo los cuatro ejercicios desarrollados en esta práctica: \texttt{string1.py}, \texttt{string2.py}, \texttt{list1.py} y \texttt{wordcount.py}. Así mismo, incluye el fichero \texttt{alice.txt}.
\end{enumerate}
\end{framed}
\end{minipage}\\\\

\section{Criterios de corrección}

La calificación de esta practica es \textit{Apto} o \textit{No Apto}
Para aprobar es necesario:
\begin{itemize}
 \item Que el resultado de TODOS los ejercicios propuestos sea correcto. El resultado es correcto cuando el mecanismo de autoevaluación incluido en cada ejercicio lo valide.
 \item Usando git haber creado y poblado el repositorio solicitado.
\end{itemize}
 En esta práctica no se valorará la calidad del código producido siempre y cuando: (1) produzca el resultado deseado, (2) 
 implemente la funcionalidad solicitada y (3) no sea un reto para un ser humano con conocimientos de Python entender el código.

\end{document}